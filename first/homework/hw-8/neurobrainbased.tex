%%%%%%%%%%%%%%%%%%%%%%%%%%%%%%%%%%%%%%%%%
% Journal Article
% LaTeX Template
% Version 1.2 (15/5/13)
%
% This template has been downloaded from:
% http://www.LaTeXTemplates.com
%
% Original author:
% Frits Wenneker (http://www.howtotex.com)
%
% License:
% CC BY-NC-SA 3.0 (http://creativecommons.org/licenses/by-nc-sa/3.0/)
%
%%%%%%%%%%%%%%%%%%%%%%%%%%%%%%%%%%%%%%%%%

%----------------------------------------------------------------------------------------
%	PACKAGES AND OTHER DOCUMENT CONFIGURATIONS
%----------------------------------------------------------------------------------------

\documentclass[twoside]{article}

\usepackage{lipsum} % Package to generate dummy text throughout this template

\usepackage[sc]{mathpazo} % Use the Palatino font
\usepackage[T1]{fontenc} % Use 8-bit encoding that has 256 glyphs
\linespread{1.05} % Line spacing - Palatino needs more space between lines
\usepackage{microtype} % Slightly tweak font spacing for aesthetics

\usepackage[hmarginratio=1:1,top=32mm,columnsep=20pt]{geometry} % Document margins
\usepackage{multicol} % Used for the two-column layout of the document
\usepackage[hang, small,labelfont=bf,up,textfont=it,up]{caption} % Custom captions under/above floats in tables or figures
\usepackage{booktabs} % Horizontal rules in tables
\usepackage{float} % Required for tables and figures in the multi-column environment - they need to be placed in specific locations with the [H] (e.g. \begin{table}[H])
\usepackage{hyperref} % For hyperlinks in the PDF

\usepackage{lettrine} % The lettrine is the first enlarged letter at the beginning of the text
\usepackage{paralist} % Used for the compactitem environment which makes bullet points with less space between them

\usepackage{abstract} % Allows abstract customization
\renewcommand{\abstractnamefont}{\normalfont\bfseries} % Set the "Abstract" text to bold
\renewcommand{\abstracttextfont}{\normalfont\small\itshape} % Set the abstract itself to small italic text

\usepackage{titlesec} % Allows customization of titles
\renewcommand\thesection{\Roman{section}}
\titleformat{\section}[block]{\large\scshape\centering}{\thesection.}{1em}{} % Change the look of the section titles

\usepackage{fancyhdr} % Headers and footers
\pagestyle{fancy} % All pages have headers and footers
\fancyhead{} % Blank out the default header
\fancyfoot{} % Blank out the default footer
\fancyhead[C]{Reading Reports $\bullet$ Alexander Moriarty $\bullet$ ISW Top30 2013} % Custom header text
\fancyfoot[RO,LE]{\thepage} % Custom footer text
\usepackage{enumitem}

%----------------------------------------------------------------------------------------
%	TITLE SECTION
%----------------------------------------------------------------------------------------

\title{\vspace{-15mm}\fontsize{24pt}{10pt}\selectfont\textbf{
Biomimetics: biologically inspired technology
}} % Article title

\author{
\large
\textsc{Yoseph Bar-Cohen}\\[2mm] % Your name
\normalsize California Institute of Technology\\ % Your institution
%\normalsize \href{mailto:john@smith.com}{john@smith.com} % Your email address
\normalsize II ECCOMAS THEMATIC CONFERENCE ON SMART STRUCTURES AND MATERIALS\\
Lisbon, Portugal, July 18-21, 2005
\vspace{-5mm}
}
\date{}

%----------------------------------------------------------------------------------------

\begin{document}

\maketitle % Insert title

\thispagestyle{fancy} % All pages have headers and footers

%----------------------------------------------------------------------------------------
%	ABSTRACT
%----------------------------------------------------------------------------------------

\begin{abstract}

\noindent Over 3.8 billion years of evolution nature introduced highly effective and power efficient 
biological mechanisms offering incredible models for innovation inspiration. Humans have always 
made efforts to imitate nature and we are increasingly reaching levels of advancement that it 
becomes significantly easier to imitate, copy, and adapt biological methods, processes and systems. 
Advances in science and technology are leading to knowledge and capabilities that are multiplying 
every year. This brought us to the ability to create technology that is far beyond the simple 
mimicking of nature. Having better tools to understand and to implement nature's principles we are 
now equipped like never before to be inspired by nature and to employ our tools in far superior 
ways. Effectively, by bio-inspiration we can have a better view and value of nature capability while 
studying its models to learn what can be extracted, copied or adapted. Using electroactive polymers 
(EAP) as artificial muscles is adding an important element in the development of biologically 
inspired technologies. This paper reviews the various aspects of the field of biomimetics and the 
role that EAP play and the outlook for its evolution. 

\end{abstract}

%----------------------------------------------------------------------------------------
%	ARTICLE CONTENTS
%----------------------------------------------------------------------------------------

\begin{multicols}{2} % Two-column layout throughout the main article text

\section{Keywords}
\begin{itemize}[noitemsep]
\item Biomimetics
\item Biologically Inspired
\item Electroactive Polymers (EAP)
\item Mimicking
\item Robotics
\item Design
\end{itemize}

%------------------------------------------------

\section{Concepts}
\begin{itemize}[noitemsep]
\item Learn from Nature
\item Efficient Design
\item Smart Structural Design
\item New Materials \& Mediums
\end{itemize}


%------------------------------------------------

\section{Thoughts \& Summary}
\begin{itemize}[noitemsep]
\item This paper nicely introduces reasons for looking at biology as a model in all areas of robotics, with emphasis on developing modern biologically inspired materials.
\item The Author mentions the need to sort `biological capabilities along technological categories.' High level categories could be: AI, Structures \& Tools, Materials and processes, Artificial Muscles, Bio-Sensors
\item Paper emphasizes the current state of the art, shortcomings and what lies ahead in developing new biologically inspired materials for use in robotics 
\end{itemize}


%------------------------------------------------
%----------------------------------------------------------------------------------------
%	REFERENCE LIST
%----------------------------------------------------------------------------------------

\begin{thebibliography}{99} % Bibliography - this is intentionally simple in this template

\bibitem[Yoseph Bar-Cohen, 2005]{BarCohen05}
Bar-Cohen, Yoseph (2005).
\newblock Biomimetics: biologically inspired technology
\newblock{\em II ECCOMAS THEMATIC CONFERENCE ON SMART STRUCTURES AND MATERIALS}
\newblock{\em C.A. Mota Soares et al. (Eds.) }
\newblock {\em Lisbon, Portugal, July 18-21, 2005}
 
\end{thebibliography}

%----------------------------------------------------------------------------------------

\end{multicols}

\end{document}
