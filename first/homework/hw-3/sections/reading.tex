\section{Gross Motion Planning}

\begin{frame}
    \frametitle{Gross Motion Planning}
    \framesubtitle{Looking at a survey article}
	\begin{itemize}
	\item Analysis of the composition and structure of Y.K.Hwang, N. Ahuja. Gross Motion Planning. ACM Computing Surveys, 24(1), pp. 220-291, 1992. 
     \item Look at explaining how the authors approached the subject.
     \end{itemize}
\end{frame}

\begin{frame}
    \frametitle{Gross Motion Planning}
    \framesubtitle{Sections}
	\begin{enumerate}
	\item Nature of Motion-Planning Problems
	\item Basic Issues \& Steps in Motion Planning
	\item Survey of Recent Work
	\item Conclusions, Acknowledgements, References
	\end{enumerate}	    
\end{frame}


\begin{frame}
    \frametitle{Gross Motion Planning}
    \framesubtitle{Structure}
    The article has been well structured. The article flows from the beginning building upon an introduction of motion planning, and defining the problem. Followed by detailing the issues and abstract steps usually taken in solving a motion planning problem. 
\end{frame}

\begin{frame}
    \frametitle{Gross Motion Planning}
    \framesubtitle{Structure}
    After introducing the problem and current methods, the paper gets into further details, surveying the recent work on Gross Motion Planning. The key here is that the paper has answered the questions, what motion planning is, why it is relevant and abstractly what others have already done. In the section `Basic Issues and Steps in Motion Planning' the paper has introduced the issues, hinting at why what has been done is not yet sufficient.   
    %% Content
\end{frame}

\begin{frame}
    \frametitle{Gross Motion Planning}
    \framesubtitle{Content}
    Now that the author, Y.K.Hwang, has introduced the topic, and essentially explained why this is yet not sufficient it is time to introduce the current research. The author takes a critical approach not only summarizing what they are doing, but why, what approaches they are using, and why might this be better. 
\end{frame}